% $Id: manual.tex,v 1.3 2003/06/23 05:32:40 hal Exp $
%
% minimalistic manual for the QM/MM-MD program
%



\documentclass[12pt,a4paper]{article}

\usepackage{a4wide}
\usepackage[tbtags]{amsmath}
\usepackage{amssymb,mathrsfs}
\usepackage{tabularx}
\usepackage{overcite}


\newcommand{\Option}[1]{\texttt{#1}}
\newcommand{\OptionValue}[1]{\texttt{#1}}
\newcommand{\Code}[1]{\texttt{#1}}
\newcommand{\CmdLine}[1]{\texttt{#1}}
\newcommand{\Latin}[1]{\textit{#1}}
\newcommand{\Program}[1]{\textsc{#1}}


\begin{document}


\section*{QM/MM-MD User Manual (for Version 2.2)}


\subsection*{Usage}

\hspace{0.1\columnwidth}\CmdLine{qmmm\_md [-O] [-h] [-w CF2\_file] [$<$]
  \itshape{input file}}


\subsection*{Description}

QM/MM-MD is a program for combined Quantum Mechanics/Molecular
Mechanics\cite{Sherwood:NIC-3-285,Maseras:JComputChem-16-9-1170,%
  Matsubara:JPhysChem-100-7-2573,Maseras:ChemCommun-25-1821,%
  Svensson:JPhysChem-100-50-19357,Dapprich:JMolStructTheoChem-461;462-1,%
  Bakowies:JPhysChem-100-25-10580,Sauer:JComputChem-21-16-1470,%
  Warshel:JMolBiol-103-227,Aqvist:ChemRev-93-7-2523,%
 Singh:JComputChem-7-6-718,Field:JComputChem-11-6-700,Gao:RevCompChem-7-119,%
 Gao:AccChemRes-29-6-298,Monard:AccChemRes-32-10-904} calculations using
Molecular Dynamics\cite{Allen:CompSimLiquid-1987,Frenkel:UnderMolSim-1996,%
  Rapaport:ArtMolDynSim-1997} by sampling in an $NVT$
ensemble\cite{McQuarrie:StatMech-1976}.  The default mode is to run only
Molecular Dynamics.  Upon user request, however, additional \Latin{ab initio}
forces within a user defined sphere---the ``hotspot''
cell\cite{Kerdcharoen:ChemPhys-211-313}---may be calculated utilizing an
external program such as
\Program{Turbomole}\cite{Ahlrichs:METECC-95,Ahlrichs:ChemPhysLett-162-3-165,%
  Brode:JComputChem-14-10-1142,Ahlrichs:BerBunsenPhysChem-102-6-795,%
  Arnim:JComputChem-19-15-1746} or \Program{Gaussian}\cite{GAUSSIAN98}.  The
total force is then calculated by the equation (subtraction
scheme\cite{Sherwood:NIC-3-285,Matsubara:JPhysChem-100-7-2573,%
  Svensson:JPhysChem-100-50-19357,Dapprich:JMolStructTheoChem-461;462-1})
\begin{equation}
  \label{eq:force}
  F_{\text{tot}} = F_{\text{MD, system}} + S (F_{\text{QM, hotspot}} -
  F_{\text{MD, hotspot}})
\end{equation}
where $F_{\text{MD, system}}$ are the MD forces of the total simulation system,
$S$ is a switching function\cite{Stillinger:JChemPhys-60-4-1545}, $F_{\text{QM,
    hotspot}}$ are the quantum mechanical forces of the hotspot cell, and
$F_{\text{MD, hotspot}}$ are the MD forces of the hotspot cell.

The input file is read either from standard input or by supplying the input
file name as the first argument to the program.  By default the output, energy,
trajectory, and velocity files (present from a previous run) are not
overwritten but can be forced to be by specifying \Code{-O}.  The optional
\Code{-w} flag specifies an alternative CF2
water\cite{Stillinger:JChemPhys-68-2-666,Jansco:ChemPhys-85-377,%
  Bopp:ChemPhysLett-98-2-129} parameter file.  If called with \Code{-h} the
program shows a short help message and terminates immediately.

The input file is in namelist format (with the name \Code{\&cntrl}) as
implemented in many popular implementations of FORTRAN 77 compilers and which
is now standard in Fortran 90 (note that the end of the namelist should be
denoted by \Code{\&end}, although some implementations might accept standard
Fortran 90's slash character '\Code{/}').

Parameters describing the potential function are input after the end of the
namelist.  The potential function follows the general form
\begin{equation}
  \label{eq:two-body}
  \mathscr{V}_{\text{2bd}} = \frac{q_1q_2}{r} + \frac{A}{r^a} + \frac{B}{r^b} +
  \frac{C}{r^c} + \frac{D}{r^d} + E\exp(-Fr)
\end{equation}
$r$ denotes the distance between pairs of atoms; $q_1$ and $q_2$ are the
charges of atoms involved; $A$, $B$, $C$, $D$, and $E$ are energy parameters,
and $F$ is the exponentials parameter; $a$, $b$, $c$, and $d$ are integer power
parameters.  Parameters are usually taken from a previous potential fit.  Two
sets, both for oxygen and hydrogen two-body interactions, must be specified.
Terms which are not used in the potential function are ``switched off'' by
setting the appropriate parameters to zero.

Additionally a three-body correction potential might be specified to augment
the two-body potential.  This is necessary especially for doubly and higher
charged ions because the pure two-body potentials are too less repulsive.
The three-body potential is currently described by
\begin{equation}
  \label{eq:three-body}
  \begin{split}
  \mathscr{V}_{\text{3bd}}^{\text{corr}}  &= \Biggl[A \exp{(-Br_{12})}
  \exp{(-Br_{13})} \exp{(-Cr_{23})} - \frac{D}{r_{12}^k} - \frac{D}{r_{13}^k} -
  \frac{E}{r_{23}^k}\Biggr] \times \\
  & \qquad (r_{\text{limit}} - r_{12})^2 (r_{\text{limit}} - r_{13})^2
  \end{split}
\end{equation}
where $r_{12}$ is the distances between ion and oxygen 1, $r_{13}$ is the
distances between ion and oxygen 2, $r_{23}$ is the distance between oxygen 1
and oxygen 2; $A$, $B$, and $C$ are potential parameters and $D$, $E$, and $k$
are optional potential parameters.  The last two terms make sure that not only
the function but also it's first derivative are zero at the cutoff limit
$r_{\text{limit}}$.  Currently the forces are evaluated only for the ion and
the two oxygens, hydrogens stay unaffected.

The program also supports umbrella sampling~\cite{Torrie:ChemPhysLett-28-578,%
  Roux:CompPhysComm-91-275} along a reaction coordinate which is currently the
distance between two atoms.  Upon user request the quadratic biasing potential
\begin{equation}
  \label{eq:umbrella}
  \mathscr{V}_{\text{bias}} = \frac{1}{2} k (r - r_0)^2
\end{equation}
is applied where $k$ is a user chosen force constant and $r_0$ the desired
distance to be sampled.  Several windows (different $r_0$'s) along the reaction
coordinate are required to get the (biased) free energy (potential of mean
force, PMF).  The $k$ and $r_0$ values for each window must be supplied at the
end of the input file each on a separate line.

In order to find the parameters for the water model in use, the environment
variable \Code{QMMM\_MD\_ROOT} must be set.  This variable points to the root
directory of the QM/MM-MD installation.  The external program which handles the
\Latin{ab initio} calculations is also searched in this directory.  The file
extension of this program may be set with the environment variable
\Code{QMMM\_MD\_FPROG}.  The default is `tm', the other currently valid values
are `gauss' and `gus' (only partially implemented).  The environment variable
\Code{QMMM\_MD\_NPROCS} may be set to the number of processors
\Program{Turbomole} should use.  The environment variable \Code{QMMM\_MD\_STMO}
may be set to `DFT' to initiate a DFT calculation intended to improve the
initial guess.


\subsection*{Control parameters}

In the following all control parameters will be discussed in two column format.
The left column contains the parameter, the right side it's description.  The
data type of each parameter is noted in parentheses in the first column just
after the parameter's name.  The letters mean: C denotes character type, I
denotes integer type, F denotes floating point type, and L denotes logical
(boolean) type.  Most of the parameters have reasonable defaults as noted at
the end of the second column in parentheses.  A few other parameters, however,
must be specified explicitly, otherwise the program will terminate with an
error.

\subsubsection*{Parameters controlling filenames of input and output files}

\begin{tabularx}{\columnwidth}{p{0.25\columnwidth}X}
  \Option{fout}(C) & output file which shows a banner, a summary of all
  parameters with their values, and energy statistics as the simulation
  proceeds (\OptionValue{QMMM-MD.out}) \\
  \Option{fenout}(C) & energy file which contains the temperature, pressure
  (currently only rudimentarily implemented), and energy contributions at each
  time step in a multicolumn format
  (\OptionValue{QMMM-MD.en}) \\
  \Option{fqmen}(C) & energy file which contains the instantaneous QM/MM
  energies (empty, only written if set) \\
  \Option{finfo}(C) & energy statistics of the last \Option{nstat} steps
  (\OptionValue{QMMM-MD.info}) \\
  \Option{fstart}(C) & file which contains the starting configuration: two sets
  of coordinates, velocities, and forces (\OptionValue{QMMM-MD.rvf}) \\
  \Option{frst}(C) & restart file, the current configuration (two sets of
  coordinates, velocities, and forces) will be written to this file
  (\OptionValue{QMMM-MD.rst}) \\
  \Option{ftraj}(C) & trajectory file (empty, only written if set) \\
  \Option{fveloc}(C) & velocity file (empty, only written if set) \\
  \Option{fumbst}(C) & histogram obtained from umbrella sampling
  (\OptionValue{umb.hst}) \\
  \Option{fumbtr}(C) & umbrella sampling trace file, writes distances as
  described by \Option{nuidx1} and \Option{nudix2} for every step (empty, only
  written if set) \\
\end{tabularx}


\subsubsection*{Parameters controlling output and frequency of output}

\begin{tabularx}{\columnwidth}{p{0.25\columnwidth}X}
  \Option{nstat}(I) & update energy statistics every \Option{nstat} steps in
  files \Option{fenout} and \Option{finfo}, also controls the correction of
  linear momentum if hard-scaling is done (\OptionValue{10}) \\
  \Option{naver}(I) & write energy statistics every \Option{naver} steps in
  file \Option{fout}, must be a multiple of \Option{nstat} to be effective
  (\OptionValue{0}) \\
  \Option{nout}(I) & energy statistics is written to the output file
  \Option{fout} every \Option{nout}$\times$\Option{nstat} steps
  (\OptionValue{1}) \\
  \Option{nrst}(I) & write restart file \Option{frst} every \Option{nrst} steps
  (\OptionValue{10}) \\
  \Option{noutcv}(I) & update the trajectory and velocity file \Option{ftraj}
  every \Option{noutcv} steps (\OptionValue{10}) \\
\end{tabularx}


\subsubsection*{Parameters controlling time steps}

\begin{tabularx}{\columnwidth}{p{0.25\columnwidth}X}
  \Option{dt}(F) & time step $\delta t$ in picoseconds (\OptionValue{0.0002})
  \\
  \Option{nstep}(I) & number of time steps to be performed (\OptionValue{10})
  \\
  \Option{init}(L) & if \Option{init} is true start counting time steps from
  zero, otherwise read time step at which to start from the restart file
  \Option{frst} (\OptionValue{t}) \\
  \Option{iprcor}(L) & if \Option{iprcor} is true the normal
  predictor/corrector algorithm is used, velocity Verlet otherwise
  (\OptionValue{t})
\end{tabularx}


\subsubsection*{Parameters controlling the temperature}

\begin{tabularx}{\columnwidth}{p{0.25\columnwidth}X}
  \Option{topt}(F) & sets the target temperature $T$ of the system in Kelvin
  (\OptionValue{298.16}) \\
  \Option{thaut}(F) & sets the relaxation time $\tau$ in picoseconds
  (\OptionValue{0.1}) \\
  \Option{nscale}(I) & hard-scale the velocities of water molecules by
  $\sqrt{T_{\text{opt}}/T}$ the first \Option{nscale} steps, the linear momenta
  are corrected every \Option{nstat} steps (\OptionValue{0}) \\
\end{tabularx}


\subsubsection*{Parameters controlling potentials and forces}

\begin{tabularx}{\columnwidth}{p{0.25\columnwidth}X}
  \Option{ihot}(L) & if \Option{ihot} is true, perform a hotspot calculation;
  Note: options \Option{ron} and \Option{roff} must be set as well
  (\OptionValue{f}) \\
  \Option{ithree}(L) & if \Option{ithree} is true, the three-body potential
  will be applied for the ion; Note: options \Option{a3bd}, \Option{b3bd}, and
  \Option{c3bd} must also be set (\OptionValue{f}) \\
  \Option{a3bd}, \Option{b3bd}, \Option{c3bd} (3F) & three-body potential
  parameters (no defaults, must be set explicitly if \Option{ithree} is true)
  \\
  \Option{d3bd}, \Option{e3bd}, \Option{k3bd} (2F, I) & optional three-body
  potential parameters, see equation~(\ref{eq:three-body}) (\OptionValue{0.0,
    0.0, 0}) \\
  \Option{iguess}(L) & if \Option{iguess} is true, the initial guess of the
  start MO vector must be supplied by the user, otherwise the external force
  program must calculate the initial guess from scratch (\OptionValue{f}) \\
  \Option{irf}(L) & if \Option{irf} is true, apply the reaction
  field\cite{vanGunsteren:FaradayDiscussChemSoc-66-58,Adams:MolPhys-38-2-387}
  (\OptionValue{t}) \\
  \Option{ion}(L) & if \Option{ion} is true, a simulation with an ion is
  performed, otherwise a simulation of neat liquid (\OptionValue{t}) \\
  \Option{intra}(L) & if \Option{intra} is true, the intra-molecular
  potential\cite{Bopp:ChemPhysLett-98-2-129} will be applied, otherwise only
  the CF2 water
  model\cite{Stillinger:JChemPhys-68-2-666,Jansco:ChemPhys-85-377}
  (\OptionValue{t}) \\
  \Option{iumb}(L) & if \Option{iumb} is true, an umbrella biasing potential
  will be applied to \Option{nuidx1} and \Option{nudix2} (\OptionValue{f}) \\
  \Option{nwin}(I) & number of windows for umbrella sampling (\OptionValue{0})
  \\
  \Option{numbeq}(I) & number of pre-equilibration steps in each window
  (\OptionValue{0}) \\
  \Option{nuidx1}, \Option{nuidx2}(2I) & indexes of atoms describing a reaction
  coordinate along which the umbrella biasing potential will be applied
  (\OptionValue{0, 0}) \\
  \Option{hdel}(F) & bin size of statistics for umbrella sampling
  (\OptionValue{0.1})
\end{tabularx}


\subsubsection*{Parameters controlling distances}

\begin{tabularx}{\columnwidth}{p{0.25\columnwidth}X}
  \Option{dens}(F) & sets the density of the solution in g/cm$^3$
  (the density of pure water at 298\,K, i.e. 0.997\,g/cm$^3$) \\
  \Option{ron}, \Option{roff}(2F) & defines the region in \AA{}ngstr\"om where
  the switching function\cite{Stillinger:JChemPhys-60-4-1545} should be applied,
  \Option{roff} also sets the size of the hotspot cell and must be grater than
  2.5\AA (no defaults, must be set explicitly if \Option{ihot} is true) \\
  \Option{rcut}(F) & sets the cutoff radii in \AA{}ngstr\"om for all Coulomb
  interactions and non-Coulomb ion-O, ion-H, and O-O interactions
  (no defaults, must be set explicitly) \\
  \Option{r3bd}(F) & sets the cutoff radius $r_{\text{limit}}$ in
  \AA{}ngstr\"om for the three-body interactions (6.0\,\AA)
\end{tabularx}


\subsubsection*{Parameters controlling properties of the ion and water}

\begin{tabularx}{\columnwidth}{p{0.25\columnwidth}X}
  \Option{ion}(L) & if \Option{ion} is true a simulation with an ion is done,
  otherwise a simulation of pure water (true) \\
  \Option{ionnam}(C) & abbreviated element name of the ion (no default, must be
  set explicitly) \\
  \Option{qion}(F) & atom charge of the ion (no default, must be set
  explicitly) \\
  \Option{amion}(F) & atom mass of the ion looked up from internal
  periodic table, but can be given explicitly to overwrite internal value \\
  \Option{amo}(F) & atom mass of the oxygen (\OptionValue{15.9994}) \\
  \Option{amh}(F) & atom mass of the hydrogen (\OptionValue{1.00794}) \\
\end{tabularx}


\subsubsection*{Potential parameters}

Potential parameters for the two-body potential are not part of the namelist,
but are input at the end of the namelist in the following format.  Six floating
point values on a line specify the parameters $A$ through $F$, on the next line
the four floating point parameters $a$ through $d$ are given.  This input is
applied twice: once for oxygen parameters and once for hydrogen parameters.  If
any parameters are not needed, refer to equation~(\ref{eq:two-body}) and set
these parameters to zero accordingly.

In contrast, three-body parameters are input as namelist variables since there
are only five of them.


\bibliographystyle{manual}
\bibliography{journal-abbrev,manual}

\end{document}


%%% Local Variables:
%%% mode: latex
%%% TeX-master: t
%%% End:

% LocalWords:  qmmm ile TURBOMOLE ntrl OOT PROG PROCS fout rudimentarily finfo
% LocalWords:  nstat fstart rvf frst rst ftraj traj fveloc vel nout nrst noutc
% LocalWords:  noutv nstep init iprcor Verlet topt thaut nscale ron ihot roff
% LocalWords:  iguess irf ithree rcut ngstr ionnam qion amion amo amh Matsubara
% LocalWords:  Turbomole JPhysChem Maseras ChemCommun Svensson Dapprich Sauer
% LocalWords:  JMolStructTheoChem Bakowies JComputChem Warshel JMolBiol Aqvist
% LocalWords:  ChemRev Singh Gao RevCompChem AccChemRes Monard Rapaport NVT PMF
% LocalWords:  ArtMolDynSim initio hotspot Brode Ahlrichs BerBunsenPhysChem gus
% LocalWords:  Arnim Bopp ChemPhysLett namelist cntrl Roux CompPhysComm FPROG
% LocalWords:  NPROCS STMO DFT fenout fqmen fumbst umb hst fumbtr nuidx nudix
% LocalWords:  naver noutcv iumb nwin numbeq pre hdel
